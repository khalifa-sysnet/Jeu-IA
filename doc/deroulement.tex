\newpage
\section{Déroulement du projet}
\label{sec:deroulement}

Dans cette section, nous décrivons comment le projet a été réalisé en équipe : la répartition des tâches, la synchronisation du travail en membres de l'équipe, etc.

\subsection{Réalisation du projet par étapes}

\subsubsection{Mise en commun des différentes idées}

\paragraph{}La mise en commun des différentes idées est une étape cruciale dans tout projet collaboratif, et en tant qu'équipe, nous avons pu participer activement à cette étape en apportant nos idées et en discutant pour trouver les solutions les plus appropriées. Dans le cadre de notre projet informatique, nous avons notamment mis en commun nos réflexions pour la conception de l'interface utilisateur, l'architecture du logiciel, les fonctionnalités à implémenter, ainsi que pour la compréhension approfondie du sujet et de ses contraintes. Cette étape a permis de définir les grandes lignes du projet et d'assurer une cohérence entre les différentes parties du logiciel.

\subsubsection{Analyse des codes fournit en genie logiciel }

\paragraph{}L'analyse des codes fournis en génie logiciel est une étape cruciale dans le développement de tout projet informatique. Dans notre cas, nous avons étudié différents codes sources fournis, notamment les projets Aircraft, Tree, Train et Chrono. Cette analyse nous a permis de comprendre la structure du code et les algorithmes utilisés pour le développement de ces mini projets.

Nous avons ainsi pu nous inspirer de ces codes pour concevoir notre propre code en nous appuyant sur les bonnes pratiques et en évitant les erreurs. Cette analyse nous a également fourni des pistes de réflexion pour le choix des algorithmes à utiliser dans notre propre projet, en nous aidant à mieux comprendre les enjeux et les défis liés au développement de notre logiciel.

\subsection{Répartition des tâches entre membres de l'équipe}

\paragraph{}Dans tout projet informatique, la répartition des tâches est une étape a ne pas negliger pour assurer la réussite du projet. En effet, il est important de bien définir les tâches à réaliser et de les attribuer aux membres de l'équipe en fonction de leurs compétences et de leur charge de travail. 






\begin{table}[H]
    \centering
    \begin{tabular}{|c|c|}
       \hline
 Tâches & Répartition  \\
\hline
 Recherche et planification & Yanis/Ilyas   \\
\hline
 Développement de l'architecture du logiciel & Yanis   \\
\hline
 Implémentation des fonctionnalités principales   & Ilyas/Yanis \\
\hline
 Documentation du projet & Ilyas/Yanis  \\
\hline
 Conception de l'interface graphique & Ilyas \\
\hline
 Création des algorithmes de la simulation & Yanis/Ilyas   \\
\hline
 Conception de la scoreboard & Ilyas \\
\hline
    \end{tabular}
    \caption{Tableau de répartition des tâches}
    \label{tab:Tableau répartition des tâches}
\end{table}

