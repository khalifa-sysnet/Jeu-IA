%%Définir le format du document: papier, taille de police, type de document, etc.

\documentclass[11pt, french]{article}

%%%%%%%%%% Packages externes utilisés %%%%%%%%%%%%%%%%%%%
\usepackage[french]{babel}
\selectlanguage{french}
\usepackage[T1]{fontenc}
\usepackage[utf8]{inputenc}

\usepackage{verbatim}
\usepackage{graphicx}
\usepackage{epstopdf}
\usepackage{macro}
\usepackage{algorithm}
\usepackage{algorithmic}
%\usepackage{algorithm2e}


%La mise en page du rapport, NE PAS MODIFIER.
\usepackage{geometry}
 \geometry{
 a4paper,
 left=20mm,
 right=20mm,
 top=20mm,
 bottom=20mm,
 }

%%%%%%%%% Le corps du document entre begin et end %%%%%%%%%%%%%%%%%%%
\begin{document}

%Page de garde
%%%%%%%%%%%%%%% Page de garde %%%%%%%%%%%%%%%%%%%

\begin{titlepage}{
    \begin{center}
        \vspace* {25mm}
        {\Large \textbf {Université de Cergy-Pontoise}} \\
        \vspace* {10mm}
        {\Large \textbf {RAPPORT}} \\
        \vspace* {10mm}
        pour le projet Génie Logiciel \\
        \textbf {Licence d'Informatique deuxième année} \\
        \vspace* {10mm}

	sur le sujet \\
        \vspace* {10mm}
	{\Huge \textsf{Gardien}} \\
        \vspace* {10mm}
 	rédigé par \\
        \vspace* {10mm}
        {\Large \textbf {BENLEKBIR Ilyas - TEL Yanis}} \\
				\vspace* {10mm}
				\noreffig{images/logo-g.png}{10.82cm}{8.2cm} \\
        \date Mars 2023
        \vspace* {10mm}
	\end{center}
}
\end{titlepage}


%Génération automatique de la table des matières, de la liste des figures et de la liste des tableaux
\tableofcontents
\listoffigures
\listoftables

%Une section "remerciements" pourrait être intéressante. C'est une section sans numérof (avec un * )


\newpage
\section{Introduction}
\label{sec:introduction}

Nous sommes le groupe GARDIEN, composé de deux membres : TEL Yanis et BENLEKBIR Ilyas. Ce projet a pour objectif de réaliser une simulation. Elle consiste simplement à ce que des gardiens doivent attraper des intrus dans un parc (que l'on va représenter sous forme de grille) dans un temps imparti.


\subsection{Contexte du projet}

\paragraph{ }Nous avons placé ce projet en troisième position dans notre liste de souhaits car il fait partie des projets sur lesquels nous souhaitions travailler. Les quelques lignes de description ont suffi pour nous donner une image claire des attentes du projet. De plus, le fait que ce projet soit une simulation et non un jeu nous a confortés dans notre choix.


\textbf{ }

\subsection{Objectif du projet}

\paragraph{ }Ce projet a pour objectif de nous familiariser avec les aspects élémentaires du génie logiciel de manière ludique. À travers cette simulation, diverses notions algorithmiques seront employées, comme par exemple la génération d'une carte aléatoire, les déplacements automatiques ou encore l'algorithme de Dijkstra (aussi appelé chemin le plus court).


\newpage
\section{Spécification du projet}
\label{sec:specification}

\paragraph{} Nous avons présenté l'objectif du projet dans la section \ref{sec:introduction}. Dans cette section, nous présenterons la spécification de notre logiciel réalisé. Ceci correspond principalement au document de spécification du projet (cahier des charges).

\subsection{Notions de base et contraintes du projet}
\label{sec:spec1}

\subsubsection{Fonctionnement général du logiciel}


\paragraph{Simulation :}
Tout d'abord, il est important de préciser que c'est une simulation et non un jeu. Cela signifie que l'utilisateur aura la possibilité de configurer la partie, mais une fois cela réalisé, il ne pourra plus interagir directement avec ce qu'il ce passera.

\paragraph{Règle :}
Nous verrons une map qui sera un parc entouré de mur, des gardien(s) ainsi que des intru(s) qui apparaitront aléatoirement dans la map.
Il y aura aussi des obstacles (eaux, arbres et rochers) qui seront aussi placés de manière aléatoirement dans la map. Au début, les gardiens ainsi que les intrus ce déplaceronts aléatoirement dans la map. Lorsqu'un intru rentre dans le champ de vision d'un gardien, le gardien va poursuivre l'intru afin de l'attraper et l'intru va fuir afin de ne pas se faire attraper par les gardiens. Au bout d'un certain moment, les gardiens communiqueront entre eux afin de se concentrer sur un seul intru.


\paragraph{Configuration :}
L'utilisateur aura la possibilité de configurer plusieurs paramètres. Il aura la possibilité de choisir le taux d'occupation d'eaux, de rochers et d'arbres sur la map. Il pourra aussi choisir le nombre de gardien, ainsi que le nombre initiale d'intru.
Il configurera aussi le temps entre les appartions automatiques d'intru ainsi que le rayon des champ de visions. Enfin, il pourra choisir à ça guise le temps entre les communications des gardiens

\subsubsection{Notions et termonologies de base}

\paragraph{ } 
\paragraph{Objet } 
Un objet est un élément présent dans une case. Ça peut être un obstacle (arbre, mur ou
eau) ou un acteur (gardien ou intrus). Il peut y avoir un ou plusieurs objet(s) pour une seule case
de la grille mais il peut aussi ne pas en avoir.

\paragraph{Obstacle } 
Dans sa définition fondamentale , un obstacle est un élément qui empêche ou gêne le
passage. Dans ce projet , nous avons trois obstacles possibles qui sont :

• Arbre : Empêche le gardien de voir au-delà de cet obstacle mais ne l'empêche pas de
passer -> on peut donc avoir plusieurs objets sur une case ou se trouve un arbre mais rend
obsolète le champ de vision du gardien.

• Eau : Empêche le gardien de passer sur cette case mais ne l'empêchent pas de voir au-
delà -> aucun autre objet ne peut être sur la même case que l’eau mais le champ de vision
reste effectif.

• Mur : Empêche le gardien à la fois de voir et de passer -> aucun autre objet ne peut être
sur la même case qu’un mur et rend obsolète le champ de vision du gardien.

Pour le nombre d'obstacles sur la map (carte), on donne la possibilité à l'utilisateur d’initialiser le
pourcentage d'obstacles sur la map sinon ce sera généré automatiquement

\paragraph{Grille} 
La grille est l'élément principal et déterminant. Il permet de situer et structurer les gardiens,
les intrus ainsi que les obstacles. Nous pourrons donner le nombre d'obstacles présents. Elle
pourra aussi être générée aléatoirement. Elle sera sous la forme d’un carré N*N, où le N sera un
1
paramètre modélisé aléatoirement (ou non) avant le lancement de la partie et est modifiable par
l’utilisateur si il en a la volonté. Informatiquement, la grille sera un graphique avec un axe x et un
axe y où on ne prendra en compte que les entiers. Il pourra y avoir (ou non) un ou plusieurs
élément(s) dans chaque coordonnées (case) du tableau/graphique.

\paragraph{Case } 
Une case est un point sur un graphique. Il aura une valeur en abscisse (x) et une valeur en
ordonnée (y). Il y aura potentiellement un ou plusieurs objet(s) (obstacle et/ou acteur) dans chaque
case. Il peut aussi ne pas en avoir, la case est dans ce cas vide. Dans cet exemple ci , la taille du
case est 1x1.

\paragraph{Déplacement d’un acteur} 
Le déplacement est un changement de position, dans notre cas, changement de position
d’une case à une case voisine de façon uniquement verticale ou horizontale (on exclura donc les
déplacements en diagonales). Cela signifie donc un changement de coordonnée: on va
ajouter/soustraire 1 au x ou au y. De ce fait, l’acteur qui aura été déplacé se trouvera sur de
nouvelles coordonnées.

Dans ce projet ci, le déplacement initiale de chaque acteur sera de une case par une case
Il n’y aura que les acteurs qui pourront se déplacer. Ils auront quatres possibilités de
déplacements: se déplacer à la case du haut, de bas, de gauche ou de droite, sauf en cas
exceptionnel lors de la rencontre d’une bordure de map ou un obstacle infranchissable (eau ou
mur). Dans ce cas là, les nouvelles possibilités de déplacement seront réduite: on va soustraire le
nombre de direction possible initial (4) par le nombre de cases voisines contenant un obstacle
infranchissable (bordure de map, mur, eau).

\paragraph{Gardiens} 
Un gardien est un objet graphique qui occupe un espace bien défini sur la grille (la map),
ce dernier se déplace dans la grille et il va avoir pour mission de repérer les intrus, en respectant
les règles déjà définies (Notion de déplacement d’un acteur) à propos de ses déplacements.
Initialement, le gardien se déplace aléatoirement d’une case à une autre jusqu'à ce qu’un intrus
rentre dans son champ de vision, il le poursuit en se déplaçant dans le chemin le plus court vers
lui. Le gardien attrape l’intrus si et seulement s'ils se retrouvent sur la même case (case gardien =
case intrus) sauf le cas où ils se croisent dans la case arbre (case arbre = case gardien = case
intrus), car la vision sera coupé pour la gardien à cause de l’arbre. Les gardiens seront entre 1 et 5
(paramètre modifiable par l'utilisateur). Si, au bout de 20 secondes (on le modifiera peut-être plus
tard), le gardien n’a toujours pas attrapé l’intrus qu’il poursuit, il communiquera la position de cet
intrus aux autres gardiens. S’il croise d'autres intrus pendant la course, il communique la position
de ce nouvel intrus aux autres gardiens. La vitesse du gardien sera 1,2x plus grande que celle de
l’intrus.

\paragraph{Scoreboard :} 
Une Scoreboard ou en français tableau des scores va résumer quelques informations
pertinentes sur le déroulement de la simulation.

Le tableau des scores sera à coté de la map (carte) qui représentera la simulation du jeu , il comportera les informations suivantes:

• Le nombre d'intru attrapé.

• Le nombre de gardiens effectif sur la carte

• Le nombre d’intrus effectif sur la carte

• Le pourcentage d'obstacle qu'occupe la map

• Le temps sous forme de chronomètre


\paragraph{Intru :} 
Un intrus est un acteur. Initialement, il se déplace de manière aléatoire, jusqu’à ce qu’un
gardien rentre dans son champ de vision. Si un gardien rentre dans son champ de vision, il va fuir
le gardien: il va se déplacer aléatoirement sans la possibilité de se déplacer dans la direction du
gardien. Par exemple: nous avons à la base 4 possibilité de déplacement pour chaque acteur qui
sont haut,bas,gauche et droite, sauf en cas exceptionnel (bordure de map, mur...). Si le gardien
apparaît à gauche de l’intrus, ces nouvelles possibilités de déplacements seront haut, bas ou
droite (neutralisant donc le déplacement gauche).

Toutes les 30 secondes, un intrus va apparaître sur une case aléatoire (excepté les cases
contenant déjà un objet),

\paragraph{Champs de vision :} 
Le champ de vision est un espace où les acteurs peuvent voir. Ici, nous avons trouvé deux
possibilité de modélisation:

Une vision angulaire qui consiste, en prenant compte de la position des acteurs ainsi que
de leurs orientations. Par exemple, si un gardien regarde vers le nord, son champ de vision
s'oriente vers le nord dans un angle de 90°.

Nous nous portons davantage sur la deuxième option qui serait de définir le champ de
vision de gardien comme étant une “aura” (un cercle) effective de n case autour du gardien. Ici, ce
sera une zone de rayon n case (qu’on choisira plus tard), où les acteurs auront accès aux
informations présentes dans ce champ. Ils pourront donc réagir en conséquence.

\paragraph{Temps (chronomètre) :} 
Le chronomètre est un instrument de mesure du temps qu’on initialise à 0 min, 0 sec. Dans
ce projet, l’unité la plus petite du temps mesuré est une seconde. Ainsi, le chronomètre à réaliser
consiste à gérer trois valeurs essentielles : heure, minute et seconde. Toutes les 30 secondes
(temps modifiable plus tard). La simulation s’arrêtera 5 minutes après le lancement ou lorsque le
compteur d’intrus sera égal à 0.

\paragraph{Coopération (Les gardiens s’aident mutuellement) :} 
Lorsqu’un gardien poursuit un intrus pendant 20 secondes, il envoie un signal à tous les
autres gardiens de la map. Si ces gardiens ne sont pas en pleine poursuite, l’intrus qui a été
signalé sera considéré comme dans leurs champs de vision afin de connaître la position de l’intrus
continuellement (pour éviter la contrainte d’avoir un signalement non pertinent, si l’on donnerait la
case de l’intrus au moment du signalement, l’intrus ne serait plus présent sur cette case à l’arrivé
des autres gardiens).

\paragraph{"Chemin le plus court” :}  
Il désigne le parcours le plus rapide pour que le gardien puisse atteindre l’intrus repéré, en
se déplaçant sur le minimum de cases franchissables pour minimiser le temps afin de maximiser la
chance de réussite pour le gardien. Cela sera réalisable en se basant sur l’algorithme Dijkstra, ce
dernier est un algorithme qui permet de trouver le plus court des chemins possibles dans la map
de jeu. Comme définition propre, Dijkstra utilise une approche de recherche en largeur pour
trouver le plus court chemin entre un sommet de départ et tous les autres sommets dans un
graphe pondéré. Il utilise une file de priorité pour stocker les sommets à explorer, et met à jour les
distances en utilisant la distance actuelle plus le poids de l'arête, mais il sera adapté à notre cas
d’utilisation.


\subsubsection{Contraintes et limitations connues}

\paragraph{} Nous avons été confrontés à une situation difficile lors de la réalisation de notre projet. Nous avions prévu d'être trois pour travailler ensemble, mais à cause de mésententes, nous nous sommes retrouvés à deux pour le mener à bien. Cette situation nous a beaucoup inquiétés et nous avons eu peur de ne pas pouvoir accomplir le projet comme prévu.

Notre peur a été exacerbée par le fait que notre seule expérience jusque-là avait été le projet POO du semestre 3, qui avec le recul était relativement simple et presque ridicule en comparaison avec ce projet. Cela a ajouté à notre anxiété et à notre sentiment de ne pas être à la hauteur du défi qui nous attendait.

Malgré cela, nous avons réussi à relever le défi et à mener à bien le projet à deux. Nous avons travaillé dur pour combler les lacunes que nous avions et nous avons finalement produit un projet dont nous sommes fiers. Cette expérience nous a appris que même dans les situations difficiles, avec du travail acharné et de la détermination, il est possible de surmonter les obstacles et de réussir. 

\paragraph{} Nous aurions aimé offrir à l'utilisateur la possibilité de contrôler un gardien dans notre projet. Nous avons envisagé cette fonctionnalité comme une option qui aurait permis aux utilisateurs de sélectionner le gardien de leur choix en cliquant dessus et de le déplacer via les touches z q s d de leur clavier. Cependant, nous avons dû faire face à des contraintes de temps et qui ne nous ont pas permis de mettre en œuvre cette fonctionnalité. Malgré cela, nous avons tout de même réussi à créer un projet qui, nous l'espérons, sera apprécié par les utilisateurs et leur permettra de découvrir une simulation passionnante.


\paragraph{Outils de développement :}
\begin{enumerate}
\item Java
\item Eclipse
\item Latex
\end{enumerate}

\subsection{Fonctionnalités attendues du projet}
\label{sec:spec2}

\paragraph{}

\paragraph{ }
L'utilisateur du logiciel doit pouvoir :

• Paramétrer la taille de la map (Choisir le nombre de case en longueur et en hauteur).

• Paramétrer le nombre de gardien

• Paramétrer le pourcentage qu'occupe les arbres sur la map

• Paramétrer le pourcentage qu'occupe l'eau sur la map

• Paramétrer le pourcentage qu'occupe les rochers sur la map

• Paramétrer le nombre d’intrus initiales.

• Ajouter un intrus au cours de la partie via un bouton

• Avoir la possibilité de mettre fin à la simulation à tout moment.

• Consulter le scoreboard (nombre d'intrus attrapé etc.)

• Distinguer les différents acteurs sur la map( via l’utilisation de code couleur ou autre...)

• Distinguer les différents obstacles sur la map (via l’utilisation de code couleur ou autre ...)

• Etre averti quand un intru entre dans le champ de vision d’un gardien.



\newpage
\section{Conception et réalisation du projet}
\label{sec:impl}

\subsection{Architecture globale du logiciel}


%\begin{figure}
%\centering
%\includegraphics[width=3.5cm, height=2cm]{images/programmer.png}
%\caption{Un programmeur occupé}
%\label{fig:modele}
%\end{figure}


\subsection{Conception des classes de données}

\paragraph{}Notre projet ce présente en différents package.

\subsubsection{Map}

\paragraph{}Ce package va nous servir à initialiser et utiliser notre map ainsi que les éléments qui s'y trouve. Nous y aurons les classes suivantes:

\paragraph{Block :} La classe a des méthodes pour obtenir la ligne et la colonne du bloc, ainsi que pour savoir si le bloc est de l'eau, un arbre, un rocher ou simplement de l'herbe. Elle a également des méthodes pour savoir si le bloc est au sommet, en bas, à gauche, à droite ou sur le bord de la carte pour délimiter le périmètre et y mettre des murs.,

\paragraph{Grille :} La classe Map est une classe qui représente une carte de jeu en utilisant des blocs, chacun étant un objet Block. Cette carte peut être créée en spécifiant un nombre de lignes et de colonnes. Les blocs sont stockés dans un tableau (une liste de liste de block).
La classe fournit des méthodes utiles pour vérifier les propriétés des blocs. Par exemple, isWall() permet de déterminer si un bloc est un mur ou non en vérifiant s'il est sur un bord de la carte. Les méthodes isWater(), isTree() et isRock() renvoient true si le bloc est respectivement de l'eau, un arbre ou un rocher. La méthode isFloor() renvoie true si le bloc est une surface de jeu normale, c'est-à-dire s'il n'est pas un mur, de l'eau, un arbre ou un rocher.
La méthode getBlockPercentage() calcule le pourcentage de blocs qui ne sont pas une surface de jeu normale. Elle parcourt tous les blocs de la carte et compte ceux qui ne sont pas une surface de jeu normale. La valeur renvoyée est un double arrondi à deux décimales.

\paragraph{Tree :} Cette classe représente un arbre dans la carte du jeu. C'est un obstacle. Elle a un attribut "position" qui est un objet Block et qui représente la position de l'arbre sur la carte. La classe a un constructeur pour initialiser l'attribut "position" et deux méthodes pour accéder à cet attribut : "getPosition" pour obtenir la position de l'arbre et "setPosition" pour définir la position de l'arbre.

\paragraph{Rock :} Cette classe représente un rocher dans la carte du jeu. C'est un obstacle. Elle a un attribut "position" qui est un objet Block et qui représente la position de l'arbre sur la carte. La classe a un constructeur pour initialiser l'attribut "position" et deux méthodes pour accéder à cet attribut : "getPosition" pour obtenir la position de l'arbre et "setPosition" pour définir la position de l'arbre.

\paragraph{Wall :} Cette classe représente un mur dans la carte du jeu. C'est un obstacle. Elle a un attribut "position" qui est un objet Block et qui représente la position de l'arbre sur la carte. La classe a un constructeur pour initialiser l'attribut "position" et deux méthodes pour accéder à cet attribut : "getPosition" pour obtenir la position de l'arbre et "setPosition" pour définir la position de l'arbre.

\paragraph{Water :} Cette classe représente de l'eau dans la carte du jeu. C'est un obstacle. Elle a un attribut "position" qui est un objet Block et qui représente la position de l'arbre sur la carte. La classe a un constructeur pour initialiser l'attribut "position" et deux méthodes pour accéder à cet attribut : "getPosition" pour obtenir la position de l'arbre et "setPosition" pour définir la position de l'arbre.

\paragraph{}

\subsubsection{Mobile}

\paragraph{}

\paragraph{Gardien :} La classe Gardien étend la classe MobileElement. Elle prend une instance de la classe Block et une instance de la classe Map comme paramètres pour son constructeur. Elle a également une variable "etatPoursuite" de type booléen, afin de differencier l'etat normal de l'etat de poursuite (notament pour son mode de déplacement).

\paragraph{Intru :} étend aussi de la classe MobileElement. Elle possède aussi un constructeur prenant en paramètres un objet Block et un objet Map.

\paragraph{Mobile Element :} La classe MobileElement est une classe abstraite qui représente un élément mobile dans le jeu. Elle possède une position représentée par un objet de la classe Block. Elle a également une méthode getZone qui prend en entrée un objet de la classe Gardien ou de la classe Intru, ainsi qu'une carte (Map), et retourne une liste de blocs représentant la zone de vision de l'élément mobile. La méthode calcule cette zone en parcourant tous les blocs de la carte et en ajoutant ceux qui se trouvent dans le rayon de vision de l'élément mobile. Les blocs qui sont ajoutés doivent être soit des sols (isFloor()), soit de l'eau (isWater()), afin qu'ils ne bloquent pas la vision de l'élément mobile.
Vous l'aurez compris, les 2 classes filles cette classe abstraite sont les classes Gardien et Intru.

\paragraph{}

\subsubsection{Images}

\paragraph{}

\paragraph{} C'est ici que nous rangerons nos images : arbre, eau, gardien, gardien-poursuite, herbe, intru, mur et roche.

\subsection{Conception des traitements (processus)}

\paragraph{}

\subsubsection{Process }

\paragraph{} Ce package va réunir toutes les classes et les méthodes qui nous permettrons d'initialiser et de faire tourner la simulation.

\paragraph{InitSimulation :} Cette classe contient des méthodes statiques pour initialiser la simulation.
La méthode "buildMap()" retourne une carte du jeu initialisée avec des obstacles tels que les murs, les arbres, les rochers et les zones de sol. Elle utilise la classe "Map" du package et les variables de configuration de la simulation pour générer les blocs.
La méthode "buildInitMobile(Map map)" retourne un "MobileElementManager" initialisé avec des gardiens et des intrus. Elle utilise la méthode "initGardien(Map map, MobileElementManager manager)" et "initIntru(Map map, MobileElementManager manager)" pour ajouter des instances de "Gardien" et "Intru" à la liste des éléments mobiles.
Les méthodes "initGardien(Map map, MobileElementManager manager)" et "initIntru(Map map, MobileElementManager manager)" initialisent un nombre spécifié de gardiens et d'intrus à des positions aléatoires sur la carte. Elles utilisent la méthode "getRandomNumber(int min, int max)" pour obtenir des nombres aléatoires.

\paragraph{MobileElementManager :} La classe gère le déplacement des gardiens et intrus dans la simulation.  La classe contient des méthodes pour déplacer les éléments vers le haut, le bas, la gauche ou la droite sur la map, ainsi que des méthodes pour les déplacer aléatoirement. La classe garde également une trace du nombre d'intrus attrapés et de la scoreBoard.
La classe prend une map dans son constructeur et deux de gardien et d'intru.
La classe a des méthodes pour déplacer chaque type d'élément mobile (Gardien et Intru) vers le haut, le bas, la gauche ou la droite. Chaque méthode de déplacement vérifie si la nouvelle position de l'élément mobile est une position valide sur la carte de jeu (c'est-à-dire si c'est un bloc d'arbre ou de sol), et si c'est le cas, met à jour la position de l'élément mobile en conséquence.
La classe a également une méthode RandomMoveGardiens qui déplace aléatoirement gardien sur la carte de jeu, idem pour les intrus.
Dans l'ensemble, la classe MobileElementManager est responsable de la gestion du mouvement des éléments mobiles dans le jeu et de la gestion des informations pertinentes telles que le score et le nombre d'intrus attrapés.

\paragraph{}

\subsubsection{GUI} Ce package va faire en sorte de structurer et d'associer les éléments que nous manipulons afin qu'on puisse les voir sur une IHM graphique.

\paragraph{}

\paragraph{BackgroundPanel :} La classe BackgroundPanel étend JPanel et sert à afficher une image de fond dans une fenêtre Swing. Elle prend en entrée le chemin d'accès à l'image en tant que chaîne de caractères lors de sa création.
Le constructeur de la classe utilise l'image spécifiée pour définir la taille préférée de la zone de dessin, puis stocke l'image en tant que variable d'instance pour une utilisation ultérieure.
La méthode paintComponent() est Override pour dessiner l'image de fond en utilisant la méthode drawImage() de l'objet Graphics. L'image commence à ce dessinée aux coordonnées (0,0) pour remplir toute la zone de dessin.

\paragraph{ConfigurerFrame :} Cette classe créee une interface utilisateur graphique pour un panneau de configuration. Ce panneau de configuration permet à l'utilisateur de configurer les paramètres d'un jeu de simulation. L'utilisateur peut entrer le nombre de lignes, de colonnes, d'intrus, de gardes, la taille de la vision, le temps de spawn, le temps de communication, la chance de bloc d'eau, la chance de bloc d'arbre et la chance de bloc de roche.
L'utilisateur peut entrer les valeurs en utilisant des champs de texte, puis cliquer sur le bouton "Confirmer choix" pour confirmer les choix. Les valeurs entrées par l'utilisateur sont stockées dans la classe GameConfiguration en tant que variables statiques. La classe est utilisée pour stocker et récupérer les valeurs des paramètres pour le jeu.
Le code crée un JFrame et ajoute plusieurs JPanels en utilisant le gestionnaire BorderLayout. Le premier JPanel contient le titre du panneau de configuration. Le deuxième JPanel contient une GridLayout de 10 lignes et 2 colonnes. Chaque ligne a un JLabel et un JTextField, représentant les étiquettes et les champs d'entrée pour les paramètres de configuration. Le troisième JPanel contient un JButton pour confirmer les choix de l'utilisateur.

\paragraph{GameDisplay :} La classe GameDisplay hérite de JPanel et est responsable de l'affichage du jeu. Elle prend en entrée  la map et un objet MobileElementManager qui sont utilisés pour récupérer les informations sur les éléments mobiles du jeu (donc les gardiens et les intrus). La méthode paintComponent est utilisée pour dessiner les éléments graphiques du jeu, y compris les Gardiens et les Intrus. Si un Gardien est en état de poursuite, il est dessiné différemment des autres Gardiens. La méthode paintVision est utilisée pour dessiner la vision de chaque Gardien et Intrus. Cette classe fournit également une méthode getManager qui renvoie l'objet MobileElementManager utilisé pour gérer les éléments mobiles du jeu.

\paragraph{GameMenu :} La classe GameMenu crée une interface utilisateur graphique pour le menu principal de jeu de la simulation. Elle crée une fenêtre JFrame contenant un BackgroundPanel avec une image de fond, un titre centré, trois boutons de taille personnalisée avec des actions associées (Jouer, Configurer et Quitter), et un JPanel pour les boutons. Cette classe permet aux utilisateurs de choisir entre jouer à la simulation, configurer les paramètres du jeu ou quitter le programme. La classe utilise des gestionnaires de mise en page pour organiser les composants et rendre l'interface utilisateur conviviale et facile à utiliser !

\paragraph{GUISec :} La classe GUISec est une classe qui hérite de la classe JFrame et implémente l'interface Runnable. Elle est utilisée pour créer une interface graphique pour un jeu.
La classe contient un certain nombre de composants graphiques, tels que des JPanel, des boutons, un tableau de bord et une carte, qui sont ajoutés à la fenêtre principale.
La méthode init() est utilisée pour initialiser les composants graphiques et les ajouter à la fenêtre principale.
La méthode run() est utilisée pour exécuter la boucle de jeu en continu (de manière iteratif), en appelant les méthodes nextRound() et repaint() de l'objet MobileElementManager et de l'objet GameDisplay, respectivement.
L'interface ActionListener est utilisée pour écouter les événements des boutons "Ajouter un intru" et "Quitter" et déclencher les actions appropriées lorsqu'ils sont cliqués.

\paragraph{MainGUI :} La classe MainGUI hérite de JFrame et implémente l'interface Runnable. Elle crée une fenêtre graphique avec une zone de jeu et un champ de texte. La zone de jeu est initialisée avec une carte et les éléments mobiles. La fenêtre peut être redimensionnée et contient un gestionnaire d'éléments mobiles. La classe utilise un thread pour mettre à jour les éléments mobiles à intervalles réguliers et redessiner la zone de jeu.

\paragraph{Paint Strategy :} Cette classe permet de dessiner la carte et les éléments du jeu.
La méthode paintMap permet de dessiner la carte. Elle prend en paramètre un objet de la classe Map et un objet de la classe Graphics qui sera utilisé pour dessiner la carte. Elle parcourt tous les blocs de la carte et dessine l'image correspondant à chaque bloc en fonction de son type (mur, eau, arbre, rocher, herbe).
La méthode paintGardien permet de dessiner le gardien. Elle prend en paramètre un objet de la classe Gardien et un objet de la classe Graphics qui sera utilisé pour dessiner le gardien. Elle récupère la position du gardien et dessine l'image du gardien à la position correspondante.
La méthode paintGardienPoursuite permet de dessiner le gardien en mode poursuite. Elle prend en paramètre un objet de la classe Gardien et un objet de la classe Graphics qui sera utilisé pour dessiner le gardien. Elle récupère la position du gardien et dessine l'image du gardien en mode poursuite à la position correspondante.
La méthode paintIntru permet de dessiner l'intru. Elle prend en paramètre un objet de la classe Intru et un objet de la classe Graphics qui sera utilisé pour dessiner l'intru. Elle récupère la position de l'intru et dessine l'image de l'intru à la position correspondante.
Enfin, la méthode paintVision permet de dessiner la zone de vision du gardien. Elle prend en paramètre un objet de la classe Gardien, un objet de la classe Graphics qui sera utilisé pour dessiner la zone de vision, et un objet de la classe Map. Elle récupère la zone de vision du gardien, calcule une transparence pour l'image de la zone de vision, et dessine un rectangle semi-transparent représentant la zone de vision autour du gardien.

\paragraph{Scoreboard :} Cette classe est pour un tableau de bord graphique qui affiche diverses statistiques de jeu, telles que le nombre d'intrus attrapés, le nombre de gardiens, le nombre d'intrus, le pourcentage d'obstacles et le temps. Il dispose également de méthodes pour mettre à jour et incrémenter les valeurs de ces statistiques.
Le tableau de bord est implémenté sous forme de JPanel avec un GridBagLayout. Il comporte différents JLabel pour afficher les statistiques et un minuteur pour mettre à jour le temps chaque seconde.
Le constructeur met en place le JPanel avec les JLabel et lance le Timer. La méthode startTimer() initialise l'heure de début et planifie une TimerTask pour mettre à jour l'heure chaque seconde. La méthode updateTime() met à jour le JLabel de temps en calculant la différence entre l'heure de début et l'heure actuelle.
Les méthodes updateintrusCaughtCount(), updateintrusCount() et updateguardiensCount() mettent à jour les JLabel correspondants avec la valeur de compte donnée. La méthode setInitialCounts() définit les comptes initiaux pour le nombre d'intrus, de gardiens et d'intrus totaux, ainsi que le pourcentage d'obstacles. Les méthodes incrementintrusCount(), decrementintrusCount(), incrementguardiensCount(), decrementguardiensCount() et incrementintrusCaughtCount() incrémentent ou décrémentent les valeurs de compte correspondantes de 1. Enfin, les méthodes setintrusCount(), setguardiensCount() et set...() définissent les valeurs de compte correspondantes à la valeur de compte donnée.
En somme, ce code fournit un tableau de bord graphique de base qui peut être utilisé pour afficher les statistiques de jeu et les mettre à jour pendant le jeu.

\paragraph{Simulation Utility :} Cette classe est utilisée pour charger une image afin d'avoir un rendu visuel.

\paragraph{}


\subsubsection{Test }

Le package Test est simplement utilisé pour lancer la simulation.

\paragraph{}

\paragraph{TestGame :} Cette classe contient une méthode main qui permettra de lancer la simulation. On va simplement instancier un GameMenu.

\paragraph{}

\subsection{Conception de l'IHM graphique}


\subsubsection{Organisation des fenêtres}


\paragraph{}L’organisation de l’IHM est une partie très important pour l’utilisateur. Plus l’IHM est simple et
agréable, plus elle sera ergonomique, ce qui va permettre à l’utilisateur de réaliser ses tâches de façon
efficace, avec un plaisir garantie. C’est pour cela qu’on a décidé de faire un menu princiapl assez simple avec
que 3 possibilitées d’interaction. La première interaction permet a l’utilisateur d’arriver sur la page de déroulement de la simulation en ne choissisant donc pas les parametres de la partie. La seconde interaction permet aux utilisateurs
de configurer sa simulation. Et la dernière interaction permet de
quitter l’application.


\begin{figure}[H]
	\centering
		\includegraphics[scale=0.4]{images/bouttons.png}
	\caption{Possibilitées de l'utilisateur }
	\label{fig:Bouttons de redirection}
\end{figure}




\paragraph{}La seule fenêtre qui peut potentiellement être complexe à comprendre est celle contenant l'affichage de la simulation. C'est pourquoi je vais la décrire.


\begin{figure}[H]
	\centering
		\includegraphics[scale=0.4]{images/orga.png}
	\caption{Organisation de la fenêtre de simulation}
	\label{fig:Organisation des fenêtres}
\end{figure}



\subsubsection{Enchainement des fenêtres}


\paragraph{}L'enchaînement des fenêtres se fait de la manière suivante : lors du lancement du jeu, on arrive sur la page d'accueil, aussi appelée menu principal, puis trois boutons nous sont proposés. Le premier permet de lancer directement la simulation et mène à la fenêtre de simulation. Le deuxième permet de configurer sa partie et mène à la fenêtre de configuration. Le troisième permet simplement de quitter le logiciel.


\begin{figure}[H]
	\centering
		\includegraphics[scale=0.4]{images/Userinterfacediagram1.png}
	\caption{Enchainement des fenêtres}
	\label{fig:Explication enchainement}
\end{figure}


\newpage
\section{Manuel utilisateur}
\label{sec:manuel}

\paragraph{}

\noindent Cette section est dédiée au manuel utilisateur. C'est la seule section dans laquelle vous pouvez utiliser des captures d'écran de votre logiciel. 


En cas de nécessité, vous pouvez scinder naturellement cette section en des sous-sections correspondantes à différentes aspects fonctionnels de votre logiciel. 

\subsection{Menu principal}

\paragraph{ }Au lancement du logiciel, c'est cette page qui nous est présentée. On peut retrouver trois boutons. En noir, le bouton "JOUER" va simplement lancer la simulation avec des paramètres par défaut. En jaune, le bouton "CONFIGURER" va permettre la redirection vers une fenêtre pour personnaliser sa simulation. En rouge, le bouton "QUITTER" permet la fermeture du logiciel.

\begin{figure}[H]
	\centering
		\includegraphics[scale=0.3]{images/principale4.png}
	\caption{Fenêtre principale}
	\label{fig:accueil}
\end{figure}

\subsection{Menu de configuration}

\paragraph{ }Comme dit précédemment, si on interagit avec le bouton "CONFIGURER", on sera dirigé vers cette fenêtre, dans laquelle on peut modifier les paramètres de la simulation : taille de la carte, nombre de gardiens, nombre d'intrus, taille du champ de vision, temps avant apparition d'intrus, temps avant communication et pourcentage de chaque obstacle sur la carte.

\begin{figure}[H]
	\centering
		\includegraphics[scale=0.3]{images/configurer.png}
	\caption{Fenêtre de configuration d'avant partie}
	\label{fig:configuration}
\end{figure}


\paragraph{ }De plus, lors de la saisie des valeurs de chaque paramètre par l'utilisateur, si une valeur est invalide, un message s'affiche pour dire à l'utilisateur que sa valeur n'est pas valide. Une fois toutes les options paramétrées, l'utilisateur peut cliquer sur le bouton "CONFIRMER" pour valider ses choix et sur le bouton "LANCER" pour démarrer la simulation.

\begin{figure}[H]
	\centering
		\includegraphics[scale=0.3]{images/message.png}
	\caption{Notification de valeur invalide}
	\label{fig:message d'erreure}
\end{figure}


\subsection{Fenêtre de le simulation}

\paragraph{ }Ceci est la fenêtre la plus importante du logiciel car celle-ci contient la simulation. Comme nous pouvons le voir, cette fenêtre est divisée en trois parties : une partie qui gère les boutons, une autre qui gère le scoreboard et la dernière qui gère le déroulement de la simulation.

\paragraph{ }Dans cette fenêtre, deux boutons au niveau inférieur sont présents : le bouton "AJOUTER UN INTRU", comme son nom l'indique, permet l'ajout en temps réel d'un intrus sur la carte, et le bouton "QUITTER" permet à l'utilisateur de fermer le logiciel à tout moment.

\paragraph{ }Le scoreboard, qui est encadré en rouge dans l'image ci-dessous, permet à l'utilisateur d'avoir des informations importantes sur la simulation en temps réel, comme le temps, le nombre de gardiens, le nombre d'intrus attrapés et le pourcentage d'obstacles sur la carte.

\begin{figure}[H]
	\centering
		\includegraphics[scale=0.3]{images/simulation2.png}
	\caption{Fenêtre du déroulement de la simulation}
	\label{fig:fenetre déroulement}
\end{figure}

\paragraph{ } Ici, nous pouvons voir plus précisément chaque information lors de la simulation :

• Encadré en rouge : ce sont les intrus.

• Encadré en bleu : ce sont les gardiens.

• Les éclairs au-dessus de la tête des gardiens signifient qu'ils sont en état de communication. En effet, dans le cercle violet, on peut voir qu'un intrus est entré dans le champ de vision d'un gardien, ce qui a mené à un état de poursuite, puis à un état de communication entre les gardiens.

• Les flèches noires montrent que les gardiens se déplacent vers l'intru qui est actuellement poursuivi par le gardien (action entourée par un cercle violet).

• Les zones jaunes représentent les champs de vision des intrus, tandis que les zones blanches représentent les champs de vision des gardiens.

• Lorsqu'un intru est attrapé, le nombre d'intrus attrapés dans la scoreboard augmente.

\begin{figure}[H]
	\centering
		\includegraphics[scale=0.3]{images/deroulement2.png}
	\caption{Explication du déroulement de la simulation}
	\label{fig:déroulement expliqué}
\end{figure}
\newpage
\section{Déroulement du projet}
\label{sec:deroulement}

Dans cette section, nous décrivons comment le projet a été réalisé en équipe : la répartition des tâches, la synchronisation du travail en membres de l'équipe, etc.

\subsection{Réalisation du projet par étapes}

\subsubsection{Mise en commun des différentes idées}

\paragraph{}La mise en commun des différentes idées est une étape cruciale dans tout projet collaboratif, et en tant qu'équipe, nous avons pu participer activement à cette étape en apportant nos idées et en discutant pour trouver les solutions les plus appropriées. Dans le cadre de notre projet informatique, nous avons notamment mis en commun nos réflexions pour la conception de l'interface utilisateur, l'architecture du logiciel, les fonctionnalités à implémenter, ainsi que pour la compréhension approfondie du sujet et de ses contraintes. Cette étape a permis de définir les grandes lignes du projet et d'assurer une cohérence entre les différentes parties du logiciel.

\subsubsection{Analyse des codes fournit en genie logiciel }

\paragraph{}L'analyse des codes fournis en génie logiciel est une étape cruciale dans le développement de tout projet informatique. Dans notre cas, nous avons étudié différents codes sources fournis, notamment les projets Aircraft, Tree, Train et Chrono. Cette analyse nous a permis de comprendre la structure du code et les algorithmes utilisés pour le développement de ces mini projets.

Nous avons ainsi pu nous inspirer de ces codes pour concevoir notre propre code en nous appuyant sur les bonnes pratiques et en évitant les erreurs. Cette analyse nous a également fourni des pistes de réflexion pour le choix des algorithmes à utiliser dans notre propre projet, en nous aidant à mieux comprendre les enjeux et les défis liés au développement de notre logiciel.

\subsection{Répartition des tâches entre membres de l'équipe}

\paragraph{}Dans tout projet informatique, la répartition des tâches est une étape a ne pas negliger pour assurer la réussite du projet. En effet, il est important de bien définir les tâches à réaliser et de les attribuer aux membres de l'équipe en fonction de leurs compétences et de leur charge de travail. 






\begin{table}[H]
    \centering
    \begin{tabular}{|c|c|}
       \hline
 Tâches & Répartition  \\
\hline
 Recherche et planification & Yanis/Ilyas   \\
\hline
 Développement de l'architecture du logiciel & Yanis   \\
\hline
 Implémentation des fonctionnalités principales   & Ilyas/Yanis \\
\hline
 Documentation du projet & Ilyas/Yanis  \\
\hline
 Conception de l'interface graphique & Ilyas \\
\hline
 Création des algorithmes de la simulation & Yanis/Ilyas   \\
\hline
 Conception de la scoreboard & Ilyas \\
\hline
    \end{tabular}
    \caption{Tableau de répartition des tâches}
    \label{tab:Tableau répartition des tâches}
\end{table}


\newpage
\section{Conclusion et perspectives}
\label{sec:conclusion}

\noindent Dans cette section, nous résumons la réalisation du projet et nous présentons également les extensions et améliorations possibles du projet.

\subsection{Résumé du travail réalisé} 

\paragraph{ }Le début du projet a été assez laborieux, notamment dû au fait que nous nous sommes retrouvés assez rapidement à deux pour un projet dans lequel nous étions initialement trois. Ceci nous a quelque peu perturbés dans la réalisation des premières tâches. Toutefois, nous sommes parvenus à nous organiser correctement afin d'accomplir ce projet pour lequel nous sommes très fiers. De la possibilité de personnaliser entièrement la carte, à la conception d'une communication entre les gardiens, en passant par un système de poursuite, jusqu'à la mise en place d'un système d'arrestation et bien d'autres encore, les aspects principaux de la simulation ont tous été réalisés. De plus, ces objectifs ont été réalisés dans les délais que nous avions convenus, et nous avons tous deux consolidé nos connaissances générales et nos méthodes de travail.


\subsection{Améliorations possibles du projet} 

\paragraph{ }Lorsque nous avions achevé les principaux aspects de la simulation, nous nous sommes rendu compte que différentes améliorations étaient possibles afin de peaufiner notre logiciel. Ces dernières sont diverses et variées, elles vont de l'ajout technique au confort visuel. Il serait possible de faire en sorte que les gardiens soient contrôlables par l'utilisateur via la simple pression d'un bouton ou encore la mise en place d'un système de "notification" qui préviendrait l'utilisateur de chaque nouvelle action sur la carte. De plus, l'aspect visuel du logiciel pourrait être amélioré en personnalisant les fenêtres avec des images en accord avec notre logiciel.

\section* {Remerciements}
Nous tenons à remercier chaleureusement Monsieur Liu Tanxiao. Votre dévouement, votre expertise et votre travail ont été essentiels à la réussite de notre projet. Chaque semaine avec vous nous ont permis d'en apprendre un peu plus, ce qui à fortement contribué à notre progression au fil des mois. Nous sommes fiers de ce que nous avons accompli grâce à vous.

\newpage
%Références bibliographiques du document

\bibliographystyle{alpha}
\bibliography{bibliographies}
\paragraph{}
• Chrono - Mr Tanxiao
\paragraph{}
• Aircraft - Mr Tanxiao
\paragraph{}
• Tree - Mr Tanxiao
\paragraph{}
• TreeV2 - Mr Tanxiao
\paragraph{}
• TreeV3 - Mr Tanxiao
\paragraph{}
• Train - Mr Tanxiao

\nocite{*}

\end{document}
